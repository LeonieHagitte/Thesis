\usepackage{booktabs}    % For table formatting
\usepackage{libertine}   % Font package (you can keep this as it is)
\usepackage{libertinust1math}  % Font package for math symbols (you can keep this as it is)
\usepackage{sourcecodepro}  % For source code font (you can keep this as it is)
\usepackage{emptypage}   % To remove the page numbers from empty pages (you can keep this as it is)
\usepackage{graphicx}    % For resizing tables to fit the text width
\usepackage{caption}     % For customizing caption alignment
\usepackage{longtable}   % For long tables spanning multiple pages
\usepackage{pdflscape}   % For landscape pages
\usepackage{rotating}    % For rotating tables (sideways tables)
\usepackage{geometry}    % For page layout customization

% Customize table settings
\renewcommand{\textfraction}{0.05}
\renewcommand{\topfraction}{0.8}
\renewcommand{\bottomfraction}{0.8}
\renewcommand{\floatpagefraction}{0.75}

% Resize tables to fit the text width and left-align captions, make them bold
\captionsetup{justification=raggedright, singlelinecheck=false,labelfont=bf}

% Optional: Change the font size for verbatim output
\let\oldverbatim\verbatim
\let\endoldverbatim\endverbatim
\renewenvironment{verbatim}{\footnotesize\oldverbatim}{\endoldverbatim}

\usepackage{pdfpages}

% titlesec does not play nice with RMarkdown, so hack:

\let\paragraph\oldparagraph
\let\subparagraph\oldsubparagraph
%\usepackage[sc, compact, raggedright]{titlesec}


