% Options for packages loaded elsewhere
\PassOptionsToPackage{unicode}{hyperref}
\PassOptionsToPackage{hyphens}{url}
%
\documentclass[
  12pt,
  a4paper,
  twoside]{article}
\usepackage{amsmath,amssymb}
\usepackage{setspace}
\usepackage{iftex}
\ifPDFTeX
  \usepackage[T1]{fontenc}
  \usepackage[utf8]{inputenc}
  \usepackage{textcomp} % provide euro and other symbols
\else % if luatex or xetex
  \usepackage{unicode-math} % this also loads fontspec
  \defaultfontfeatures{Scale=MatchLowercase}
  \defaultfontfeatures[\rmfamily]{Ligatures=TeX,Scale=1}
\fi
\usepackage{lmodern}
\ifPDFTeX\else
  % xetex/luatex font selection
\fi
% Use upquote if available, for straight quotes in verbatim environments
\IfFileExists{upquote.sty}{\usepackage{upquote}}{}
\IfFileExists{microtype.sty}{% use microtype if available
  \usepackage[]{microtype}
  \UseMicrotypeSet[protrusion]{basicmath} % disable protrusion for tt fonts
}{}
\makeatletter
\@ifundefined{KOMAClassName}{% if non-KOMA class
  \IfFileExists{parskip.sty}{%
    \usepackage{parskip}
  }{% else
    \setlength{\parindent}{0pt}
    \setlength{\parskip}{6pt plus 2pt minus 1pt}}
}{% if KOMA class
  \KOMAoptions{parskip=half}}
\makeatother
\usepackage{xcolor}
\usepackage[left=3.9cm, right=3.3cm, top=2.5cm, bottom=3cm]{geometry}
\usepackage{longtable,booktabs,array}
\usepackage{calc} % for calculating minipage widths
% Correct order of tables after \paragraph or \subparagraph
\usepackage{etoolbox}
\makeatletter
\patchcmd\longtable{\par}{\if@noskipsec\mbox{}\fi\par}{}{}
\makeatother
% Allow footnotes in longtable head/foot
\IfFileExists{footnotehyper.sty}{\usepackage{footnotehyper}}{\usepackage{footnote}}
\makesavenoteenv{longtable}
\usepackage{graphicx}
\makeatletter
\def\maxwidth{\ifdim\Gin@nat@width>\linewidth\linewidth\else\Gin@nat@width\fi}
\def\maxheight{\ifdim\Gin@nat@height>\textheight\textheight\else\Gin@nat@height\fi}
\makeatother
% Scale images if necessary, so that they will not overflow the page
% margins by default, and it is still possible to overwrite the defaults
% using explicit options in \includegraphics[width, height, ...]{}
\setkeys{Gin}{width=\maxwidth,height=\maxheight,keepaspectratio}
% Set default figure placement to htbp
\makeatletter
\def\fps@figure{htbp}
\makeatother
\setlength{\emergencystretch}{3em} % prevent overfull lines
\providecommand{\tightlist}{%
  \setlength{\itemsep}{0pt}\setlength{\parskip}{0pt}}
\setcounter{secnumdepth}{5}
\newlength{\cslhangindent}
\setlength{\cslhangindent}{1.5em}
\newlength{\csllabelwidth}
\setlength{\csllabelwidth}{3em}
\newlength{\cslentryspacingunit} % times entry-spacing
\setlength{\cslentryspacingunit}{\parskip}
\newenvironment{CSLReferences}[2] % #1 hanging-ident, #2 entry spacing
 {% don't indent paragraphs
  \setlength{\parindent}{0pt}
  % turn on hanging indent if param 1 is 1
  \ifodd #1
  \let\oldpar\par
  \def\par{\hangindent=\cslhangindent\oldpar}
  \fi
  % set entry spacing
  \setlength{\parskip}{#2\cslentryspacingunit}
 }%
 {}
\usepackage{calc}
\newcommand{\CSLBlock}[1]{#1\hfill\break}
\newcommand{\CSLLeftMargin}[1]{\parbox[t]{\csllabelwidth}{#1}}
\newcommand{\CSLRightInline}[1]{\parbox[t]{\linewidth - \csllabelwidth}{#1}\break}
\newcommand{\CSLIndent}[1]{\hspace{\cslhangindent}#1}
\usepackage{booktabs}
\usepackage{libertine}
\usepackage{libertinust1math}
\usepackage{sourcecodepro}
\usepackage{emptypage}
\renewcommand{\textfraction}{0.05}
\renewcommand{\topfraction}{0.8}
\renewcommand{\bottomfraction}{0.8}
\renewcommand{\floatpagefraction}{0.75}
%\DefineVerbatimEnvironment{Highlighting}{Verbatim}{commandchars=\\\{\},fontsize=\footnotesize}
%% change fontsize of output
\let\oldverbatim\verbatim
\let\endoldverbatim\endverbatim
\renewenvironment{verbatim}{\footnotesize\oldverbatim}{\endoldverbatim}
\ifLuaTeX
  \usepackage{selnolig}  % disable illegal ligatures
\fi
\IfFileExists{bookmark.sty}{\usepackage{bookmark}}{\usepackage{hyperref}}
\IfFileExists{xurl.sty}{\usepackage{xurl}}{} % add URL line breaks if available
\urlstyle{same}
\hypersetup{
  pdftitle={Development of a German Instrument for Self-Perceived Data Literacy},
  pdfauthor={Leonie Hagitte},
  hidelinks,
  pdfcreator={LaTeX via pandoc}}

\title{Development of a German Instrument for Self-Perceived Data Literacy}
\usepackage{etoolbox}
\makeatletter
\providecommand{\subtitle}[1]{% add subtitle to \maketitle
  \apptocmd{\@title}{\par {\large #1 \par}}{}{}
}
\makeatother
\subtitle{An Algorithm-based Approach to Scale Development}
\author{Leonie Hagitte}
\date{2024-03-16}

\begin{document}
\maketitle

{
\setcounter{tocdepth}{2}
\tableofcontents
}
\setstretch{1.2}
\newpage\null\thispagestyle{empty}\newpage

\hypertarget{abstract}{%
\section*{Abstract}\label{abstract}}
\addcontentsline{toc}{section}{Abstract}

The increasing relevance of competent and critical handling of data in society not only makes it possible to record this competence, but also makes self-perception with regard to this competence increasingly clear. Previous approaches consider this competence primarily against the specific background of individual target groups, jobs or roles (\protect\hyperlink{ref-Cui2023}{Cui et al., 2023}). In addition, only a few explicitly refer to the general population (\protect\hyperlink{ref-Carmi2020}{Carmi et al., 2020}; \protect\hyperlink{ref-Cui2023}{Cui et al., 2023}). In view of the various theoretical approaches, there is a need for a uniform definition of data literacy in order to create comparability.
Our aim is therefore to derive a holistic definition based on these approaches and to develop a questionnaire for self-perception of one's own data literacy. To this end, the decisive factors for the construct from previous definitions and operationalizations in various disciplines are brought together. Cognitive interviews are conducted iteratively to create and refine the items. The items are then selected using algorithm-based item selection. The facets of data literacy are comprehensively tested for factorial, discriminant, convergent and congruent incremental validity in order to promote a differentiated understanding of the construct. Construct and criterion validity are tested using correlations and hierarchical regression analyses, while cross-validation checks the robustness of the instrument.
Based on a cross-sectional online questionnaire study, we first examine a representative sample of people from the general population. Limitations arise from the cross-sectional design and the heuristic item reduction, which limit predictions of predictive validity. The heterogeneous nature of the construct makes global instrument development and understanding of all participants difficult.
The self-assessment questionnaire promotes a holistic assessment of competence and its perception for further research, for example by comparing self-assessment and actual performance.

\hypertarget{acknowledgements}{%
\section*{Acknowledgements}\label{acknowledgements}}
\addcontentsline{toc}{section}{Acknowledgements}

I dedicate this thesis to

I want to thank my advisers, Prof.~Martin Schultze, Prof.~Timo Lorenz, and Prof.~Manuel Völkle for their time and patience, and my friends for their resourceful advice:

\newpage\null\thispagestyle{empty}\newpage

\hypertarget{intro}{%
\section{Intro}\label{intro}}

\hypertarget{background}{%
\section{Background}\label{background}}

The relevance of data literacy in today's society becomes evident as it serves as a potent tool in navigating the complex data-driven environment. In a world characterized by information overload and rapid technological advancements, individuals equipped with strong data literacy skills can discern patterns, critically evaluate information, and make informed decisions

The exploration of citizens' interaction with media and the cultivation of their agency has traditionally centered around concepts such as written literacy, media literacy, information literacy, and digital literacy. In more recent discussions, Data Literacy has been approaching relevance among discussed competencies regarding what is necessary for agency in the current society (\protect\hyperlink{ref-Carmi2020}{Carmi et al., 2020}). Deficiency in data literacy not only exposes individuals to various risks and harms on personal, social, physical, and financial levels but also constrains their capacity to actively engage as informed citizens within an evolving, data-driven society \{Carmi et al. (\protect\hyperlink{ref-Carmi2020}{2020})\}. Thus, Data Literacy is a competency that is becoming increasingly important to everyone. And research has acknowledged this in recent years, as more and more research is being done in that direction (\protect\hyperlink{ref-Cui2023}{Cui et al., 2023}); And in praxis there are Training programmes and Workshops sprouting to enhance ones Data Literacy as well (QUELLE).

In ``Thinking, Fast and Slow,'' Kahneman (\protect\hyperlink{ref-kahneman2011}{2011}) introduces the two systems of thinking: System 1 and System 2. System 1 operates swiftly, relying on intuition and often succumbing to biases, while System 2 operates more deliberately, engaging in analytical thinking. Kahneman's exploration extends to the realization that even experts across diverse domains can fall prey to cognitive biases, leading to errors in judgment (\protect\hyperlink{ref-kahneman2011}{Kahneman, 2011}).
Applying Kahneman's insights to the realm of data literacy and research questions sheds light on a critical aspect. It underscores the notion that individuals, irrespective of their statistical expertise, remain vulnerable to cognitive biases when interpreting and analyzing data. In the current societal landscape, where data plays an increasingly pivotal role in shaping decisions and policies, acknowledging and addressing these cognitive biases becomes paramount.

We noticed several gaps in the current research that we want to try addressing with this study. For once the target groups for Data Literacy are currently each having their own definition of the construct it seems (\protect\hyperlink{ref-Cui2023}{Cui et al., 2023}). This not only makes comparisons impossible but also makes a general understanding of the topic as well as communication in the community and science communication to the public harder.

\hypertarget{data-and-information}{%
\subsection{Data and Information}\label{data-and-information}}

When speaking of Data Literacy, the respective self-perception and how to define and measure both constructs, one naturally has to think about what data is. What is information, how are they different to each other and what extents do they share? And what role does the process of interpretation play? Does one interpret data to make sense of it? And if so, does that mean that data is entropy while information stands in opposition to it?

The concepts of data and information are foundational in various fields, yet their precise definitions and relationships are often subject to interpretation.
According to Shannon's seminal work on information theory (\protect\hyperlink{ref-shannon1948}{Shannon, 1948}), data can be understood as raw, unprocessed symbols or observations, devoid of inherent meaning. It is through a process of interpretation and organization that data transforms into information, as elucidated by Bates (\protect\hyperlink{ref-bates2005}{2005}). Bates emphasizes that information emerges when data is structured and presented in a way that is comprehensible and relevant to a particular context or purpose.

Entropy, a concept borrowed from thermodynamics and applied in information theory, plays a crucial role in understanding the relationship between data and information. In his landmark paper, Shannon (\protect\hyperlink{ref-shannon1948}{1948}) defines entropy as a measure of uncertainty or disorder in a system. In the realm of information theory, entropy is often associated with the amount of unpredictability or randomness in a set of data.
However, it's essential to note that entropy can also be viewed as a measure of information content within a system. This perspective is articulated by Brillouin (\protect\hyperlink{ref-brillouin1953}{1953}), who suggests that low entropy corresponds to a high concentration of meaningful information. Similarly, the work of Jaynes (\protect\hyperlink{ref-jaynes1957}{1957}) highlights the connection between entropy and information, proposing that information can be quantified in terms of the reduction of uncertainty or entropy in a system.
Thus, we can refine our understanding of the relationship between data, entropy, and information. While data serves as the raw material from which information is derived, it's the reduction of entropy through organization and interpretation that gives rise to meaningful information. Thus, rather than viewing data as synonymous with entropy, it's more accurate to consider information as emerging from the structured representation of data, leading to a deeper understanding of the underlying phenomena.

\hypertarget{measuring-self-perceived-data-literacy}{%
\subsection{Measuring Self-Perceived Data Literacy}\label{measuring-self-perceived-data-literacy}}

instances have underscored the extent to which citizens may lack awareness regarding the potential uses and misuses of their data. {[}argumentation for selfrating measure!{]}

\hypertarget{conceptual-integration}{%
\subsection{Conceptual Integration}\label{conceptual-integration}}

finding common ground of the existing theories and

\hypertarget{understand-and-interpret-data}{%
\subsection{Understand and Interpret Data}\label{understand-and-interpret-data}}

\hypertarget{contextualize-evaluate-and-critique-data}{%
\subsection{Contextualize, Evaluate and Critique Data}\label{contextualize-evaluate-and-critique-data}}

\hypertarget{integrate-data-in-ones-view-of-the-world}{%
\subsection{Integrate Data in One's View of the World}\label{integrate-data-in-ones-view-of-the-world}}

\hypertarget{presentvisualize-datainformation}{%
\subsection{Present/Visualize Data/Information}\label{presentvisualize-datainformation}}

\hypertarget{managecollectanalyze-data}{%
\subsection{Manage/Collect/Analyze Data}\label{managecollectanalyze-data}}

\hypertarget{moral-attentiveness}{%
\subsection{Moral Attentiveness}\label{moral-attentiveness}}

\begin{itemize}
\tightlist
\item
  moral disengagement scale
\end{itemize}

Moral Attentiveness (\protect\hyperlink{ref-reynolds2008}{Reynolds, 2008}) is a concept that delves into the inherent differences individuals display when it comes to engaging with ethical considerations. Some individuals naturally exhibit a genuine interest and attentiveness to ethical matters, while others may find the subject mundane or remain indifferent, and in extreme cases, appear unresponsive. This inclination toward or against ethics appears to be deeply ingrained, persisting across diverse contexts and proving resistant to various ethics interventions, whether constructive or detrimental.
Importantly, acknowledging one's interest in ethics does not imply inherent moral virtue or vice, nor does it suggest that behavior is rigid and unchangeable. Rather, it highlights an observation familiar to many practitioners --- that certain individuals inherently pay more attention to moral matters. These individuals contribute to making organizational interventions more interactive, enjoyable, and presumably more beneficial compared to those less attuned to such ethical considerations.
Within the field, an implicit assumption is that the variance in individual moral attentiveness can be explained by two existing constructs: moral awareness and moral sensitivity. Moral awareness involves an individual's recognition that a situation possesses moral implications, while moral sensitivity refers to their capacity to achieve such moral awareness. In essence, the unspoken argument is that an individual's level of attention to moral matters can be attributed to their moral awareness, moral sensitivity, or a combination of both.

\hypertarget{critical-thinking}{%
\subsection{Critical Thinking}\label{critical-thinking}}

The conceptualization of Critical Thinking (CrT) has evolved along three main branches: philosophical, psychological, and educational (Rear, 2019). In the philosophical view, which centers on the mental process of thought, a critical thinker is someone adept at logically evaluating and questioning both the assumptions of others and their own. On the psychological front, which delves into the processes driving action, a critical thinker possesses a combination of skills enabling them to assess a situation and determine the most appropriate course of action. The educational approach aligns more closely with the psychological perspective, relying on frameworks and learning activities tailored to enhance students' CrT skills and subsequently assess their proficiency in these skills (\protect\hyperlink{ref-payan2022}{Payan Carreira et al., 2022}).

\begin{itemize}
\item
  Rear, D. One size fits all? The limitations of standardised assessment in critical thinking. Assess. Eval. High. Educ. 2019, 44, 664--675.
\item
  Thaiposri, P.; Wannapiroon, P. Enhancing Students' Critical Thinking Skills through Teaching and Learning by Inquiry-basedLearning Activities Using Social Network and Cloud Computing. Procedia-Soc. Behav. Sci. 2015, 174, 2137--2144
\end{itemize}

\hypertarget{need-for-cognition}{%
\subsection{Need for Cognition}\label{need-for-cognition}}

The personality trait known as Need for Cognition (NFC) originated in social psychology during the 1940s and 1950s, the concept of NFC, representing an inclination for joyful thinking, is evident in the works of Maslow (1943), Murphy (1947), Asch (1952), and Sarnoff and Katz (1954). The conceptualization of NFC underwent refinement in the mid-1950s through experimental investigations by Cohen and colleagues (e.g., Cohen, Stotland and Wolfe, 1955). They defined NFC as ``a need to structure relevant situations in meaningful, integrated ways. It is a need to understand and make reasonable the experiential world'' (Cohen et al., 1955, p.~291). The concept captures individual variations in the engagement and enjoyment of thinking tasks (Bless, Wänke, Bohner, Fellhauer and Schwarz, 1994).

\hypertarget{core-self-evaluation}{%
\subsection{Core self evaluation}\label{core-self-evaluation}}

-\textgreater{} core self evaluation judge, lock and durham 97
-\textgreater{} heilman und jonas 2010 validation german

Core Self-Evaluations (CSE) is a comprehensive personality trait that encompasses fundamental appraisals of one's worthiness, effectiveness, and capability as a person (Judge, Erez, Bono, \& Thoresen, 2003). This construct, initially introduced by Judge, Locke, and Durham (1997), integrates four traits: self-esteem, emotional stability, generalized self-efficacy, and locus of control, all of which contribute to individuals' perceptions of themselves and their agency (Judge \& Bono, 2001).

The study by Chang et al.~(2012) further emphasizes the role of CSE in individuals' assessments of their own worthiness and capabilities. These assessments, rooted in self-esteem, emotional stability, generalized self-efficacy, and locus of control, form the core components of CSE (Judge et al., 2003). Self-esteem, as established by Rosenberg (1965), provides a foundation for positive self-perceptions, while emotional stability (low neuroticism) characterized by confidence and steadiness (Goldberg, 1990) enhances individuals' resilience. Generalized self-efficacy, reflecting beliefs about one's ability to perform across diverse situations (Locke, McClear, \& Knight, 1996), and locus of control, which pertains to individuals' beliefs about the causes of events in their lives (Rotter, 1966), further shape individuals' perceptions of their capabilities and control over their environment.

\hypertarget{openness-towards-technology}{%
\subsection{Openness towards technology}\label{openness-towards-technology}}

The widespread integration of information and communication technology (ICT) in various aspects of daily life, notably accelerated by the COVID-19 pandemic lockdown (Richter and Mohr, 2020; Rizun and Strzelecki, 2020), has underscored the significance of individuals' self-perceived ICT competence. This self-perception extends beyond a general assessment to encompass specific competence domains, referred to as ICT self-concept (ICT-SC). ICT-SC reflects individuals' mental representations and evaluations of their ICT competences shaped by experiences, feedback, and interactions with the environment.

Research findings indicate that when the objective demands of competence encompass multiple dimensions, the corresponding self-concepts are likewise structured in a multidimensional fashion (Brunner et al., 2010). Examining the European Digital Competence Framework for Citizens (DigComp) and its updated versions, DigComp 2.0/2.1 (Carretero et al., 2017; Vuorikari et al., 2016), reveals a comprehensive structure. This framework not only integrates previous models of ICT competence but also extends its scope, incorporating models such as the Canadian Digital Skill Framework (Chinien and Boutin, 2011) and components of ICT literacy by Katz (2007), among others (e.g., Eshet-Alkalai, 2004; Ferrari, 2012; Martin and Grudziecki, 2006).
Within DigComp and its subsequent versions, ICT competence is systematically organized across five competence domains: information and data literacy, communication and collaboration, digital content creation, safety, and problem-solving. These domains encompass a total of 21 competences, providing a comprehensive and nuanced perspective on the multifaceted nature of ICT competence.

\hypertarget{methods}{%
\section{Methods}\label{methods}}

\hypertarget{sample}{%
\subsection{Sample}\label{sample}}

The sample for this study comprised XXX participants (M=, SD=). Within the sample, XXX\% identified as female, XXX\% as male, and xxx\% did not identify with binary gender categories. All participants were aged 18 and above. Regarding education, all participants exhibited a {[}insert educational level- specifying the range or types of educational levels observed in the sample{]}. Among the participants, n= reported higher knowledge on items x, x, x, leading to their selection for an additional set of items as a preliminary survey for factors four and five.
The study encompassed every sector within the occupational classification (Bundesagentur für Arbeit, 2020), ensuring comprehensive representation. Conducted in German, the participation in the study was entirely voluntary, with no external incentives provided. The recruitment of participants was carried out through a combination of personal and professional networks, along with outreach on various online social media platforms.

Our study sample serves as a focal point for comparison against the demographic landscape of the general public in Germany. In 2022, the mean age of the German population was 44.6 years, with 45,457,000 individuals engaged in employment. Educational backgrounds varied (XXX), and for gender distribution, the split was nearly 50/50 (41,616,473 males and 42,816,197 females) according to the Statistisches Bundesamt (source: \url{https://www.destatis.de/DE/Themen/Gesellschaft-Umwelt/Bevoelkerung/Bevoelkerungsstand/Tabellen/liste-zensus-geschlecht-staatsangehoerigkeit.html\#651186}).

Analyzing our sample against these benchmarks provides a comprehensive understanding of any distinctions or parallels in age, employment, education, and gender. This comparison enhances the applicability of our findings to the broader German population.

\hypertarget{preregistration}{%
\subsection{Preregistration}\label{preregistration}}

\hypertarget{instruments}{%
\subsection{Instruments}\label{instruments}}

\hypertarget{measuring-moral-attentiveness}{%
\subsection{Measuring Moral Attentiveness}\label{measuring-moral-attentiveness}}

To assess moral attentiveness, we employed the perceptual moral attentiveness scale derived from the German Moral Attentiveness Scale by Pohling et al.(2014); adapted from Reynolds, 2008). This subscale comprises four items, including statements such as ``I am regularly confronted with decisions that have significant ethical consequences.'' The scale comprises a seven-point Likert-type scale with fully labeled options, ranging from strongly disagree (1) to strongly agree (7).

\hypertarget{measuring-critical-thinking}{%
\subsection{Measuring Critical Thinking}\label{measuring-critical-thinking}}

\hypertarget{measuring-need-for-cognition}{%
\subsection{Measuring Need for Cognition}\label{measuring-need-for-cognition}}

\hypertarget{measuring-self-efficacy}{%
\subsection{Measuring Self-Efficacy}\label{measuring-self-efficacy}}

\hypertarget{measuring-openness-towards-technology}{%
\subsection{Measuring Openness towards technology}\label{measuring-openness-towards-technology}}

ICT-SC was measured, using the ICT-SC25g. The ICT-SC25 is a self-administered questionnaire designed to assess ICT self-concept (ICT-SC) on both a general scale (items 1-5) and a domain-specific scale (items 6-25). Responses to items are provided on a six-point Likert-type scale with fully labeled options, ranging from strongly disagree (1) to strongly agree (6). The questionnaire has undergone validation in German (ICT-SC25g). Its applicability extends across the adult population, spanning ages 18 to 69, and diverse contexts including work, private life, and education.

\hypertarget{item-creation}{%
\subsection{Item Creation}\label{item-creation}}

After drafting a set of items, those items were being reviewed in cognitive interviews in an iterative manner. The idea was to go through the items with different people and ask them if the items are comprehensive, whether it is clear to them what is being asked with the items and what everyone associates with the item and so on. After every interview, the remarks made by the Person interviewed get worked into the items and then the reworked item set is presented in the next interview and so on.

Berücksichtigung von z.b. Trennschärfe etc. item schwierigkeit

\hypertarget{analysis}{%
\section{Analysis}\label{analysis}}

We employed an automated item selection algorithm to craft the {[}insert name{]} scale. The process of scale development, involving the strategic selection of items to ensure psychometric soundness, is conceptualized as a combinatorial problem (Kerber et al., 2022). Combinatorial problems, exemplified by the knapsack problem (Schroeders et al., 2016), entail identifying a discrete and finite solution within predefined constraints (Hoos and Stützle, 2005).

Contemporary approaches to address these combinatorial problems leverage automatic optimization algorithms, such as Genetic Algorithms (GA; Holland, 1975), Ant Colonization Algorithms (ACO; XXX), brute force (XXX), or random sampling(XXX). (Schultze, 2022).
Unlike classical approaches that consider items based on their individual merits, heuristic item selection algorithms aim to enhance the psychometric properties of a set of items within predetermined constraints (Schultze, 2017).
Noteworthy is the inherent approximate, rather than deterministic, nature of metaheuristics (Schultze \& Lorenz ,2023; Blum and Roli, 2003). Wich makes brute force approaches the preferred choice, if applicable (Schultze \& Lorenz ,2023). Nevertheless, as brute force often isnt fesable because of timely and computational costs, approximate algorithms are indispensable for obtaining near-optimal solutions to complex combinatorial problems in a timely or computationally efficient manner (Schultze \& Lorenz ,2023; Dorigo and Stützle, 2010).

residuals
correlates or the residuals for the adjacent constructs

\hypertarget{results}{%
\section{Results}\label{results}}

\hypertarget{discussion}{%
\section{Discussion}\label{discussion}}

\begin{itemize}
\tightlist
\item
  what to optimize the scale for?
\item
  dynamic fit indices
\item
  factors 4 and 5

  \begin{itemize}
  \tightlist
  \item
    adaptive testing/ IRT

    \begin{itemize}
    \tightlist
    \item
      dimensionality assumption and computationally intense
    \end{itemize}
  \item
    CART - tree based adaptive testing (classification trees) always binary split

    \begin{itemize}
    \tightlist
    \item
      gini index to identify the cut off
    \item
      POMP method - for differing number of answer formats
    \end{itemize}
  \end{itemize}
\item
  residuals correlates
\item
  heterogeneity of construct
\end{itemize}

\hypertarget{references}{%
\section*{References}\label{references}}
\addcontentsline{toc}{section}{References}

\hypertarget{refs}{}
\begin{CSLReferences}{1}{0}
\leavevmode\vadjust pre{\hypertarget{ref-bates2005}{}}%
Bates, M. J. (2005). An introduction to metatheories, theories, and models. In M. J. Bates \& M. N. Maack (Eds.), \emph{Encyclopedia of library and information sciences} (2nd ed., pp. 109--121). Taylor \& Francis.

\leavevmode\vadjust pre{\hypertarget{ref-brillouin1953}{}}%
Brillouin, L. (1953). Negentropy principle of information. \emph{Journal of Applied Physics}, \emph{24}(9), 1152--1163.

\leavevmode\vadjust pre{\hypertarget{ref-Carmi2020}{}}%
Carmi, E., Yates, S. J., Lockley, E., \& Pawluczuk, A. (2020). Data citizenship: Rethinking data literacy in the age of disinformation, misinformation, and malinformation. \emph{Internet Policy Review}, \emph{9}(2). \url{https://doi.org/10.14763/2020.2.1481}

\leavevmode\vadjust pre{\hypertarget{ref-Cui2023}{}}%
Cui, Y., Chen, F., Lutsyk, A., Leighton, J., \& Cutumisu, M. (2023). Data literacy assessments: A systematic literature review. \emph{Assessment in Education: Principles, Policy \& Practice}, \emph{30}, 1--21. \url{https://doi.org/10.1080/0969594X.2023.2182737}

\leavevmode\vadjust pre{\hypertarget{ref-jaynes1957}{}}%
Jaynes, E. T. (1957). Information theory and statistical mechanics. \emph{Physical Review}, \emph{106}(4), 620--630.

\leavevmode\vadjust pre{\hypertarget{ref-kahneman2011}{}}%
Kahneman, D. (2011). \emph{Thinking, fast and slow}. Farrar, Straus; Giroux.

\leavevmode\vadjust pre{\hypertarget{ref-payan2022}{}}%
Payan Carreira, R., Sacau-Fontenla, A., Rebelo, H., Sebastião, L., \& Pnevmatikos, D. (2022). Development and validation of a critical thinking assessment-scale short form. \emph{Education Sciences}, \emph{12}, 938. \url{https://doi.org/10.3390/educsci12120938}

\leavevmode\vadjust pre{\hypertarget{ref-reynolds2008}{}}%
Reynolds, S. J. (2008). Moral attentiveness: Who pays attention to the moral aspects of life? \emph{Journal of Applied Psychology}, \emph{93}(5), 1027--1041. \url{https://doi.org/10.1037/0021-9010.93.5.1027}

\leavevmode\vadjust pre{\hypertarget{ref-shannon1948}{}}%
Shannon, C. E. (1948). A mathematical theory of communication. \emph{Bell System Technical Journal}, \emph{27}(3), 379--423.

\end{CSLReferences}

\end{document}
